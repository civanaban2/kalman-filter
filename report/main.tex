\documentclass[12pt,a4paper]{article}
\usepackage[utf8]{inputenc}
\usepackage[T1]{fontenc}
\usepackage[turkish]{babel}  % Turkish language support
\usepackage{amsmath}
\usepackage{amsfonts}
\usepackage{amssymb}
\usepackage{amsthm}
\usepackage{graphicx}
\usepackage{geometry}
\usepackage{listings}
\usepackage{xcolor}
\usepackage[
    colorlinks=true,
    linkcolor=black,
    citecolor=black,
    filecolor=black,
    urlcolor=blue,
    hidelinks
]{hyperref}
\usepackage{float}
\usepackage{subcaption}
\usepackage{bm}
\usepackage{cite}

\geometry{margin=2.5cm}

% Turkish theorem environments (using babel-turkish)
\newtheorem{teorem}{Teorem}
\newtheorem{lemma}{Lemma}
\newtheorem{tanim}{Tan{\i}m}
\newtheorem{onerme}{\"{O}nerme}

% Code listing settings
\lstset{
    language=C,
    basicstyle=\ttfamily\footnotesize,
    keywordstyle=\color{blue},
    commentstyle=\color{green!60!black},
    stringstyle=\color{red},
    numbers=left,
    numberstyle=\tiny\color{gray},
    frame=single,
    breaklines=true,
    captionpos=b,
    title=\lstname
}

\title{
    \LARGE \textbf{Pasif Sens\"{o}r ile Gemi Takibi} \\
    \large MTH 407 - Otomatik Hedef \.{I}zleme Temelleri
}

\author{
    \textbf{Civan Rumet Ar{\i}} \and \textbf{Yaren Sakarya} \\
}

\date{13 Haziran 2025}

\begin{document}

\maketitle

\begin{abstract}
Bu rapor, \c{c}oklu sens\"{o}r kullanarak sadece a\c{c}{\i} \"{o}l\c{c}\"{u}mleri ile gemi takibi yapan bir Geni\c{s}letilmi\c{s} Kalman Filtresi (EKF) uygulamas{\i}n{\i} sunmaktad{\i}r. Proje, ger\c{c}ek zamanl{\i} sens\"{o}r sim\"{u}lasyonu, Do\u{g}rusal Olmayan En K\"{u}\c{c}\"{u}k Kareler (NLS) ile ba\c{s}lang{\i}\c{c} tahmin ve EKF ile konum takibi algoritmalar{\i}n{\i} i\c{c}ermektedir. Sistem C dilinde geli\c{s}tirilmi\c{s} olup, bearing \"{o}l\c{c}\"{u}mlerini dosyadan okuyarak ger\c{c}ek zamanl{\i} i\c{s}leme yapmaktad{\i}r. Gauss-Newton optimizasyonu ile ba\c{s}lang{\i}\c{c} konumu belirlenmekte, ard{\i}ndan EKF predict/update d\"{o}ng\"{u}s\"{u} ile hedefin hareketi takip edilmektedir. Sonu\c{c}lar MATLAB ile g\"{o}rselle\c{s}tirilmekte ve sistem performans{\i} analiz edilmektedir.

\end{abstract}

\tableofcontents
\newpage

\section{Giri\c{s}}

\subsection{Proje Amac{\i}}
Pasif sens\"{o}r sistemleri, hedefleri sadece al{\i}nan sinyaller \"{u}zerinden tespit etmeye olanak sa\u{g}layan \"{o}nemli bir teknoloji alan{\i}d{\i}r. Bu projede, \c{c}oklu bearing (a\c{c}{\i}) sens\"{o}r\"{u} kullanarak denizde hareket eden bir geminin konumunu ve h{\i}z{\i}n{\i} ger\c{c}ek zamanl{\i} olarak tahmin eden bir sistem geli\c{s}tirilmi\c{s}tir.

\subsection{Problem Tan{\i}m{\i}}
Sadece a\c{c}{\i} \"{o}l\c{c}\"{u}mleri ile konum belirleme, do\u{g}rusal olmayan bir tahmin problemidir. N sens\"{o}rden al{\i}nan $\theta_i(t)$ a\c{c}{\i} \"{o}l\c{c}\"{u}mleri kullan{\i}larak, hedefin durum vekt\"{o}r\"{u} $\mathbf{x}(t) = [x(t), y(t), v_x(t), v_y(t)]^T$ tahmin edilmelidir. Bu vekt\"{o}r hedefin konum ve h{\i}z bilgilerini i\c{c}ermektedir.

\subsection{Yakla\c{s}{\i}m}
\begin{itemize}
    \item İlk üç ölçüm ile Gauss-Newton optimizasyonu kullanarak başlangıç konumu belirleme
    \item Genişletilmiş Kalman Filtresi ile gerçek zamanlı konum ve hız takibi
    \item C dilinde geliştirilen gerçek zamanlı simülasyon sistemi
    \item MATLAB ile sonuç analizi ve görselleştirme
\end{itemize}

\section{Matematiksel Model}

\subsection{Durum Vektörü ve Hareket Modeli}

\subsubsection{Durum Uzayı Gösterimi}
Hedefin durumu 4 boyutlu bir vektör ile tanımlanır:
\begin{equation}
\mathbf{x}(t) = [x(t), y(t), v_x(t), v_y(t)]^T
\end{equation}

\subsubsection{Sürekli Zaman Dinamik Modeli}
Hedefin hareket dinamiği sabit hız modeli ile tanımlanır:
\begin{equation}
\frac{d\mathbf{x}(t)}{dt} = \mathbf{A}\mathbf{x}(t) + \mathbf{w}(t)
\end{equation}

burada sistem matrisi:
\begin{equation}
\mathbf{A} = \begin{bmatrix}
0 & 0 & 1 & 0 \\
0 & 0 & 0 & 1 \\
0 & 0 & 0 & 0 \\
0 & 0 & 0 & 0
\end{bmatrix}
\end{equation}

\subsubsection{Ayrık Zaman Durum Geçiş Matrisi}
Örnekleme periyodu $\Delta t$ için durum geçiş matrisi:
\begin{equation}
\mathbf{F} = e^{\mathbf{A}\Delta t} = \begin{bmatrix}
1 & 0 & \Delta t & 0 \\
0 & 1 & 0 & \Delta t \\
0 & 0 & 1 & 0 \\
0 & 0 & 0 & 1
\end{bmatrix}
\end{equation}

\subsubsection{Süreç Gürültüsü Kovaryans Matrisi}
Süreç gürültüsü kovaryans matrisi $\mathbf{Q}$, sabit hız modeli için:
\begin{equation}
\mathbf{Q} = \begin{bmatrix}
\frac{\Delta t^3}{3} & 0 & \frac{\Delta t^2}{2} & 0 \\
0 & \frac{\Delta t^3}{3} & 0 & \frac{\Delta t^2}{2} \\
\frac{\Delta t^2}{2} & 0 & \Delta t & 0 \\
0 & \frac{\Delta t^2}{2} & 0 & \Delta t
\end{bmatrix} q
\end{equation}
burada $q$ süreç gürültüsü yoğunluğu parametresidir.

\subsection{Doğrusal Olmayan Ölçüm Modeli}

\subsubsection{Bearing Ölçüm Fonksiyonu}
$i$-inci sensör için, $\mathbf{s}_i = [s_{x,i}, s_{y,i}]^T$ konumunda bulunan sensörün bearing ölçümü:
\begin{equation}
z_i = h_i(\mathbf{x}) + v_i = \arctan\left(\frac{y - s_{y,i}}{x - s_{x,i}}\right) + v_i
\label{eq:bearing_measurement}
\end{equation}

burada $v_i \sim \mathcal{N}(0, \sigma_i^2)$ ölçüm gürültüsüdür.

\subsubsection{Ölçüm Jacobian Matrisi}
Ölçüm fonksiyonunun Jacobian matrisi:
\begin{equation}
\mathbf{H}_i = \frac{\partial h_i}{\partial \mathbf{x}} = \frac{1}{r_i^2} \begin{bmatrix}
-(y - s_{y,i}) & (x - s_{x,i}) & 0 & 0
\end{bmatrix}
\end{equation}

burada $r_i^2 = (x - s_{x,i})^2 + (y - s_{y,i})^2$ sensöre olan karesel mesafedir.

\section{Doğrusal Olmayan En Küçük Kareler ile Başlangıç Tahmini}

\subsection{Problem Formülasyonu}
İlk üç bearing ölçümü $\{z_1, z_2, z_3\}$ kullanılarak başlangıç konumu $\mathbf{p}_0 = [x_0, y_0]^T$ belirlenir. Optimizasyon problemi:
\begin{equation}
\min_{\mathbf{p}} \sum_{i=1}^{3} \left[ z_i - \arctan\left(\frac{y - s_{y,i}}{x - s_{x,i}}\right) \right]^2
\label{eq:nls_objective}
\end{equation}

\subsection{Gauss-Newton Algoritması}

\subsubsection{Residual Vektörü}
Residual fonksiyonu:
\begin{equation}
\mathbf{r}(\mathbf{p}) = \begin{bmatrix}
z_1 - h_1(\mathbf{p}) \\
z_2 - h_2(\mathbf{p}) \\
z_3 - h_3(\mathbf{p})
\end{bmatrix}
\end{equation}

\subsubsection{Jacobian Matrisi}
Residual fonksiyonunun Jacobian matrisi:
\begin{equation}
\mathbf{J}(\mathbf{p}) = \begin{bmatrix}
-\frac{\partial h_1}{\partial x} & -\frac{\partial h_1}{\partial y} \\
-\frac{\partial h_2}{\partial x} & -\frac{\partial h_2}{\partial y} \\
-\frac{\partial h_3}{\partial x} & -\frac{\partial h_3}{\partial y}
\end{bmatrix}
\end{equation}

\subsubsection{Gauss-Newton Güncelleme Adımı}
İteratif güncelleme kuralı:
\begin{equation}
\mathbf{p}^{(k+1)} = \mathbf{p}^{(k)} - \left(\mathbf{J}^T\mathbf{J}\right)^{-1} \mathbf{J}^T \mathbf{r}(\mathbf{p}^{(k)})
\end{equation}

Algoritma yakınsama kriteri $\|\Delta \mathbf{p}\| < \epsilon$ sağlanana kadar devam eder.

\section{Genişletilmiş Kalman Filtresi Uygulaması}

\subsection{Predict Adımı}
Durum tahmini:
\begin{align}
\hat{\mathbf{x}}_{k|k-1} &= \mathbf{F}_k \hat{\mathbf{x}}_{k-1|k-1} \\
\mathbf{P}_{k|k-1} &= \mathbf{F}_k \mathbf{P}_{k-1|k-1} \mathbf{F}_k^T + \mathbf{Q}_k
\end{align}

\subsection{Update Adımı}
Her yeni bearing ölçümü için:
\begin{align}
\mathbf{y}_k &= z_k - h(\hat{\mathbf{x}}_{k|k-1}) \\
\mathbf{S}_k &= \mathbf{H}_k \mathbf{P}_{k|k-1} \mathbf{H}_k^T + R_k \\
\mathbf{K}_k &= \mathbf{P}_{k|k-1} \mathbf{H}_k^T \mathbf{S}_k^{-1} \\
\hat{\mathbf{x}}_{k|k} &= \hat{\mathbf{x}}_{k|k-1} + \mathbf{K}_k \mathbf{y}_k \\
\mathbf{P}_{k|k} &= (\mathbf{I} - \mathbf{K}_k \mathbf{H}_k) \mathbf{P}_{k|k-1}
\end{align}

\subsection{Zaman Mantığı}
\begin{itemize}
    \item İlk 3 ölçüm: Gauss-Newton ile başlangıç tahmini
    \item 4. ölçümden itibaren: EKF predict/update döngüsü
    \item Her yeni ölçüm için predict adımı ($\Delta t$ kullanılarak)
    \item Ardından update adımı (bearing ölçümü ile)
\end{itemize}

\section{Simülasyon Kurulumu}

\subsection{Gerçek Gemi Hareketi}
\begin{itemize}
    \item Toplam zaman adımı: $N = 50$
    \item Örnekleme periyodu: $\Delta t = 1$ saniye
    \item Hareket modeli: Sabit hız ile doğrusal hareket
    \item Başlangıç konumu: $(x_0, y_0) = (100, 200)$ metre
    \item Başlangıç hızı: $(v_x, v_y) = (2, 1)$ metre/saniye
\end{itemize}

\subsection{Sensör Düzeni}
Üç adet bearing sensörü aşağıdaki konumlarda yerleştirilmiştir:
\begin{align}
\mathbf{s}_1 &= [0, 0]^T \\
\mathbf{s}_2 &= [300, 0]^T \\
\mathbf{s}_3 &= [150, 300]^T
\end{align}

\subsection{Radar Tarama Açıklaması}
\begin{itemize}
    \item Her sensör belirli aralıklarla bearing ölçümü yapar
    \item Ölçüm gürültüsü: $\sigma = 0.1$ radyan
    \item Gerçek zamanlı dosya işleme: 100ms polling periyodu
    \item Yeni ölçüm geldiğinde EKF güncelleme
\end{itemize}

\section{Sonuçlar ve Görselleştirme}

\subsection{Trajektori Grafiği}
\begin{figure}[H]
    \centering
    % Placeholder for trajectory plot
    \fbox{\parbox{12cm}{
        \centering
        \vspace{3cm}
        Trajektori grafiği buraya eklenecek \\
        (Gerçek yol vs Tahmin edilen yol)
        \vspace{3cm}
    }}
    \caption{Gerçek ve tahmin edilen gemi trajektorisi}
    \label{fig:trajectory}
\end{figure}

\subsection{RMSE Analizi}
\begin{figure}[H]
    \centering
    % Placeholder for RMSE plot
    \fbox{\parbox{12cm}{
        \centering
        \vspace{3cm}
        RMSE grafiği buraya eklenecek \\
        (Zaman vs Konum hatası)
        \vspace{3cm}
    }}
    \caption{Kök ortalama kare hata (RMSE) analizi}
    \label{fig:rmse}
\end{figure}

\subsection{Q Matrisi ve Sensör Geometrisinin Etkisi}

\subsubsection{Süreç Gürültüsü Parametresi Analizi}
\begin{table}[H]
\centering
\caption{Farklı Q değerleri için performans karşılaştırması}
\begin{tabular}{|c|c|c|}
\hline
\textbf{q değeri} & \textbf{Ortalama RMSE (m)} & \textbf{Yakınsama Süresi (s)} \\
\hline
0.1 & - & - \\
1.0 & - & - \\
10.0 & - & - \\
\hline
\end{tabular}
\label{tab:q_analysis}
\end{table}

\subsubsection{Sensör Geometrisi Etkisi}
\begin{figure}[H]
    \centering
    % Placeholder for sensor geometry effect
    \fbox{\parbox{12cm}{
        \centering
        \vspace{3cm}
        Sensör geometrisi etkisi grafiği \\
        buraya eklenecek
        \vspace{3cm}
    }}
    \caption{Sensör düzeninin tahmin performansına etkisi}
    \label{fig:sensor_geometry}
\end{figure}

\section{Tartışma}

\subsection{Seyrek Ölçümler Altında Performans}
\begin{itemize}
    \item Bearing-only takip doğası gereği gözlemlenebilirlik sorunları yaşar
    \item İlk üç ölçüm kritik öneme sahiptir - triangulation için minimum gereksinim
    \item Sensör geometrisi performansı önemli ölçüde etkiler
    \item Gauss-Newton başlangıç tahmini EKF yakınsaması için kritiktir
\end{itemize}

\subsection{Radar Periyot Duyarlılığı}
\begin{itemize}
    \item Çok sık ölçüm: İşlem yükünü artırır, marjinal performans iyileşmesi
    \item Çok seyrek ölçüm: Tahmin belirsizliğini artırır, hedef kaybı riski
    \item Optimal ölçüm periyodu hedef hızı ve sistem gereksinimlerine bağlıdır
    \item Gerçek zamanlı işleme kabiliyeti sistem tasarımında önemlidir
\end{itemize}

\subsection{Algoritma Kararlılığı}
\begin{itemize}
    \item EKF doğrusallaştırma hatalarına duyarlıdır
    \item Jacobian matrisi singularite yakınında sayısal sorunlar yaşanabilir
    \item Kovaryans matrisi pozitif tanımlılığı korunmalıdır
    \item Ölçüm gürültüsü parametreleri sistem performansını önemli ölçüde etkiler
\end{itemize}

\section{Sonuç}

\subsection{Proje Başarıları}
Bu projede başarıyla gerçekleştirilen önemli başlıklar:
\begin{enumerate}
    \item Bearing-only gemi takibi için kapsamlı bir EKF sistemi geliştirildi
    \item Gauss-Newton optimizasyonu ile etkili başlangıç tahmin algoritması uygulandı
    \item Gerçek zamanlı dosya işleme ve simülasyon sistemi oluşturuldu
    \item C dilinde yüksek performanslı matematik kütüphanesi geliştirildi
    \item MATLAB ile kapsamlı analiz ve görselleştirme araçları hazırlandı
\end{enumerate}

\subsection{Teknik Katkılar}
\begin{itemize}
    \item Doğrusal olmayan tahmin problemine matematiksel yaklaşım
    \item Gerçek zamanlı sinyal işleme sistem mimarisi
    \item Çoklu sensör füzyon algoritması
    \item Sayısal optimizasyon ve filtreleme tekniklerinin uygulaması
\end{itemize}

\subsection{Gelecek Çalışmalar}
\begin{itemize}
    \item Unscented Kalman Filter (UKF) ile performans karşılaştırması
    \item Çoklu hedef takibi ve veri ilişkilendirme
    \item Uyarlamalı gürültü tahmini algoritmaları
    \item Dağıtık sensör ağları için genişletme
    \item Makine öğrenmesi destekli başlangıç tahmin yöntemleri
\end{itemize}

\bibliographystyle{IEEEtran}
\begin{thebibliography}{10}

\bibitem{kalman1960}
R. E. Kalman, ``A new approach to linear filtering and prediction problems,'' \emph{Journal of Basic Engineering}, vol. 82, no. 1, pp. 35--45, 1960.

\bibitem{julier1997}
S. J. Julier and J. K. Uhlmann, ``New extension of the Kalman filter to nonlinear systems,'' in \emph{Signal Processing, Sensor Fusion, and Target Recognition VI}, vol. 3068, pp. 182--193, 1997.

\bibitem{aidala1979}
V. J. Aidala and S. E. Hammel, ``Utilization of modified polar coordinates for bearings-only tracking,'' \emph{IEEE Transactions on Automatic Control}, vol. 28, no. 3, pp. 283--294, 1983.

\bibitem{nardone1984}
S. C. Nardone, A. G. Lindgren, and K. F. Gong, ``Fundamental properties and performance of conventional bearings-only target motion analysis,'' \emph{IEEE Transactions on Automatic Control}, vol. 29, no. 9, pp. 775--787, 1984.

\bibitem{simon2006}
D. Simon, \emph{Optimal State Estimation: Kalman, H-infinity, and Nonlinear Approaches}. Wiley, 2006.

\end{thebibliography}

\newpage
\appendix

\section{Kod Uygulaması}

\subsection{Ana EKF Algoritması}
\begin{lstlisting}[caption=Genişletilmiş Kalman Filtresi Uygulaması]
void ekf_predict(kalman_t *kalman, double dt)
{
    // Durum tahmini: x = F * x
    get_F_matrix(dt, kalman);
    for (int i = 0; i < 4; i++) {
        double x_new = 0;
        for (int j = 0; j < 4; j++)
            x_new += kalman->F[i][j] * kalman->x[j];
        kalman->x_pred[i] = x_new;
    }
    
    // Kovaryans tahmini: P = F * P * F^T + Q
    matrix_multiply_4x4(kalman->F, kalman->P, temp_FP);
    matrix_transpose_4x4(kalman->F, F_transpose);
    matrix_multiply_4x4(temp_FP, F_transpose, temp_FPFT);
    matrix_add_4x4(temp_FPFT, kalman->Q, kalman->P);
}
\end{lstlisting}

\subsection{Gauss-Newton Optimizasyonu}
\begin{lstlisting}[caption=Gauss-Newton Başlangıç Tahmini]
void gauss_newton(kalman_t *kalman, measurement_t *measurements)
{
    gauss_newton_t gn;
    gauss_init(&gn);
    
    // Ba\c{s}lang{\i}\c{c} tahmini
    gn.x = kalman->x[0];
    gn.y = kalman->x[1];
    
    for (int iter = 0; iter < gn.max_iter; iter++) {
        // Jacobian ve residual hesaplama
        get_H_matrix(&gn, kalman, measurements);
        
        // Normal denklemler: (H^T * H) * delta = H^T * f
        double HtH[2][2], Htf[2];
        HtH_and_Htf(HtH, Htf, &gn);
        get_inv(&gn, HtH);
        
        // G\"{u}ncelleme
        gn.delta_x = gn.inv[0][0] * Htf[0] + gn.inv[0][1] * Htf[1];
        gn.delta_y = gn.inv[1][0] * Htf[0] + gn.inv[1][1] * Htf[1];
        
        gn.x -= gn.delta_x;
        gn.y -= gn.delta_y;
        
        // Yak{\i}nsama kontrol\"{u}
        if (fabs(gn.delta_x) + fabs(gn.delta_y) < gn.tol)
            break;
    }
    
    kalman->x[0] = gn.x;
    kalman->x[1] = gn.y;
}
\end{lstlisting}

\end{document}
